\documentclass{article}
\usepackage[margin=0.5in]{geometry}
\usepackage{amsmath,amssymb}
\usepackage{array,booktabs,makecell}
\usepackage{hyperref}

\title{gh3d2m parameter file definitions}
\author{}
\date{\today}
 
\begin{document}
 
\maketitle
 
%=============================================================================
\section*{Introduction}
 
Definitions for parameter files for the \texttt{gh3d2m}
numiercal relativity code.
See \cite{Pretorius:2004jg} for a general description of the code,
and \cite{Pretorius:2006tp} for a description of binary black hole
evolution with the code.

\newpage
%=============================================================================
\section*{General definitions}
\begin{table}[h]
   \centering 
   \begin{tabular}{ccc}
      Parameter  & Allowed values & Purpose \\
     \midrule\midrule
      lambda
      &
      $\left[0,1\right]$ 
      & 
      CFL number $\Delta t = \lambda \Delta x$
      \\ \\
      steps 
      &
      $\mathbb{N}$
      &
      Number of time steps to take.
      \\ \\
      tn\_eps\_diss 
      &
      $\left[0,1\right]$
      & Kreiss-Oliger dissipation parameter
      \\ \\
      diss\_use\_6th\_order 
      &
      $\left\{0,1\right\}$
      &
      (Do not) use $6^{th}$ order Kreiss-Oliger dissipation stencils.
      \\ \\
      cp\_restart 
      &
      $\left\{0,1\right\}$
      &
      (Do not) restart from a checkpoint file
      \\ \\
      cp\_restore\_fname 
      &
      \texttt{string}
      &
      Starting name of checkpoint file.
      \\ \\
      cp\_delta\_t\_hrs 
      &
      $\mathbb{R}_+$
      &
      Checkpoint after this number of hours
      \\ \\
      cp\_save\_fname 
      &
      \texttt{string}
      &
      Name of checkpoint file to save. 
      \\ \\
      use\_AH\_as\_FS 
      &
      $\mathbb{N}$
      &
      ??
      \\ \\
      FS\_radius\{\_2\}
      &
      $\mathbb{R}_+$
      &
      ??
      \\ \\
      ej\_rmax 
      &
      $\mathbb{R}_+$
      &
      ??
   \end{tabular}
\end{table}

\newpage
%=============================================================================
\section*{Output parameters}
\begin{table}[h]
   \centering 
   \begin{tabular}{ccc}
      Parameter  & Allowed values & Purpose \\
      \midrule\midrule
      save\_tag 
      &
      \texttt{string}
      &
      File save beginning.
      \\ \\
      save\_\{1,2\}\_vars &
      Vector $\left[\mathtt{string}\right]$
      &
      Variables to save from level 1,2
      \\ \\
      save\_ivec\{\_3\} 
      &
      $1-*/n,n\in\mathbb{N}$
      &
      Save every $n$ time steps
      \\ \\
      save\_all\_finer 
      &
      $\left\{0,1\right\}$
      &
      ??
      \\ \\
      save\_iter\_mod 
      &
      $\left\{0,1\right\}$
      &
      ??
   \end{tabular}
\end{table}

\newpage
%=============================================================================
\section*{Psi4 variables}
\begin{table}[h]
   \centering 
   \begin{tabular}{ccc}
      Parameter  & Allowed values & Purpose \\
      \midrule\midrule
      num\_Psi4S 
      &
      $\mathbb{N}$
      &
      ??
      \\ \\
      Psi4S\_radius0 
      &
      $\mathbb{R}_+$
      &
      ??
      \\ \\
      Psi4S\_radius1 
      &
      $\mathbb{R}_+$
      &
      ??
      \\ \\
      Psi4S\_N\{theta,phi\}
      &
      $2^n+1,n\in\mathbb{N}$
      &
      Number of points in theta, phi angular direction. 
      \\ \\
      Psi4S\_Lmin 
      &
      $\mathbb{N}$
      &
      ??
   \end{tabular}
\end{table}

\newpage

\newpage
%=============================================================================
\section*{AMR grid parameters}
\begin{table}[h]
   \centering 
   \begin{tabular}{ccc}
      Parameter  & Allowed values & Purpose \\
      \midrule\midrule
      base\_shape 
      &
      $\left[
         \mathtt{int} \;
         \mathtt{int} \;
         \mathtt{int}
      \right]$
      &
      Number of points of base grid.
      \\ \\ 
      base\_bbox 
      &
      $\left[
         \mathtt{int} \; \mathtt{int} \;
         \mathtt{int} \; \mathtt{int} \;
         \mathtt{int} \; \mathtt{int} \;
      \right]$
      &
      Bounding box of the base grid (xmin, xmax, ymin, ymax, zmin, zmax).
      \\ \\
      max\_lev 
      &
      $\mathbb{N}$
      &
      Maximum number of AMR levels
      \\ \\
      init\_depth
      &
      $\mathbb{N}$
      &
      Initial number of AMR levels.
      \\ \\
      init\_bbox\_\{n\}
      &
      $\left[
         \mathtt{int} \; \mathtt{int} \;
         \mathtt{int} \; \mathtt{int} \;
         \mathtt{int} \; \mathtt{int} \;
      \right]$
      &
      Bounding box coordinates for the $n^{th}$
      initial AMR level.
      \\
      & & 
      If you specify $n$ twice, then there are initiall two
      \\
      & & 
      different AMR grids at that level.
   \end{tabular}
\end{table}

\newpage
%=============================================================================
\section*{Gauge/formulation definitions}

\begin{table}[h]
   \centering 
   \begin{tabular}{ccc}
      Parameter  & Allowed values & Purpose \\
     \midrule\midrule
      harmonize
      &
      \{0,1\}
      & (Do not) set the time derivative of the lapse
      and shift so that $\Box x^a=0$.
      \\ \\
      a\_tilde
      &
      $\left[0,1\right]$
      &
      Modified generalized harmonic parameter; see
      \cite{East:2020hgw}.
      \\
      & & 
      Sets $\tilde{A}$ parameter in 
      $\tilde{g}^{ab} = g^{ab} - \tilde{A} n^an^b$
      \\ \\
      a\_ghat 
      &
      $\left[0,1\right]$
      &
      Modified generalized harmonic parameter; see
      \cite{East:2020hgw}.
      \\
      & & 
      Sets $\hat{A}$ parameter in 
      $\hat{g}^{ab} = g^{ab} - \hat{A} n^an^b$
      \\ \\
      kappa\_cd 
      & 
      $\mathbb{R}_-$
      &
      Constraint damping parameter $\kappa$
      \\ \\
      kappa\_cd\_cosn 
      &
      $\mathbb{N}$
      &
      Asymptotic falloff of the damping parameter (usually: 4)
      \\ \\
      rho\_cd 
      &
      $\mathbb{R}_+$
      &
      Constraint damping parameter $\rho$
   \end{tabular}
\end{table}

\newpage
%=============================================================================
\section*{Truncation error tagging parameters}

\begin{table}[h]
   \centering 
   \begin{tabular}{ccc}
      Parameter  & Allowed values & Purpose \\
      \midrule\midrule
      TRE\_max &
      $\mathbb{R}_+$
      &
      Maximum truncation error before tag for regridding.
      \\ \\
      TRE\_vars 
      &
      Vector $\left[ \mathtt{string} \right]$
      &
      Gridfunctions to sum over for tagging
      \\ \\
      TRE\_buffer
      &
      $\mathbb{N}$
      &
      Buffer size between tagging zones(?)
      \\ \\
      TRE\_ibc\_\{\_a\}\_buffer 
      &
      $\mathbb{N}$
      &
      ??
      \\ \\
      TRE\_exc\_buffer\{\_lmin\}
      &
      $\mathbb{N}$
      &
      Number of points around excision surface that are not considered
      \\
      & & 
      for truncation error tagging (as excision surface can be ``noisy'').
      \\ \\
      TRE\_ibcp\_buffer
      &
      $\mathbb{N}$
      &
      ??
      \\ \\
      TRE\_sgpbh
      &
      $\mathbb{N}$
      &
      ??
      \\ \\
      TRE\_norm
      &
      $\mathbb{N}$
      &
      ??
      \\ \\
      regrid\_interval 
      &
      $\mathbb{N}$
      &
      Regrid every number of time steps. 
      \\ \\
      regrid\_min\_lev &
      $\mathbb{N}$
      &
      Regrid on at least this level
      \\ \\
      regrid\_script := 2
      &
      $\mathbb{N}$
      &
      ??
      \\ \\
      regrid\_script\_name 
      &
      \texttt{string}
      &
      ??
   \end{tabular}
\end{table}

\newpage
%=============================================================================
\section*{Apparent horizon search parameters}

\begin{table}[h]
   \centering 
   \begin{tabular}{ccc}
      Parameter  & Allowed values & Purpose \\
      \midrule\midrule
      AH\_N\{theta,phi\}\{\_2,\_3\} 
      &
      $2^n+1$
      &
      Number of grid points in theta/phi direction for search algorithm
      \\
      & & for black hole 1,2,3
      \\ \\
      AH\_Lmin\{\_3\} 
      &
      $\mathbb{N}$
      &
      ??
      \\ \\
      AH\_Lmax 
      &
      $\mathbb{N}$
      &
      ??
      \\ \\
      AH\_max\_iter\{\_2,3\} 
      &
      $\mathbb{N}$
      &
      ??
      \\ \\
      AH\_freq\{\_aft,\_3,\_aft\_3\}
      &
      $\mathbb{N}$
      &
      ??
      \\ \\
      AH\_tol\{\_aft,\_3,\_aft\_3\}
      &
      $\mathbb{R}_+$
      &
      Tolerance to solve for apparent horizon.
      \\ \\
      AH\_max\_tol\_inc 
      &
      $\mathbb{N}$
      &
      ??
      \\ \\
      AH\_reset\_scale 
      &
      $\mathbb{R}$
      &
      ??
      \\ \\
      AH\_r0 
      &
      $\mathbb{R}$
      &
      ??
      \\ \\
      AH\_r1 
      &
      $\mathbb{R}$
      &
      ??
      \\ \\
      AH\_rsteps &
      $\mathbb{N}$
      &
      ??
      \\ \\
      AH\_lambda\{\_min,\_3,\_min\_3\}
      &
      $\mathbb{R}_+$
      &
      CFL number for apparent horizon search algorithm.
      \\ \\
      AH\_maxinc 
      &
      $\mathbb{N}$
      &
      ??
      \\ \\
      AH\_eps 
      &
      1.5
      &
      ??
      \\ \\
      AH\_xc &
      Vector: $\left[ 
         \mathbb{R} \; \mathbb{R} \; \mathbb{R} 
      \right]$
      &
      ??
      \\ \\ 
      AH\_tmin\{\_3\} 
      &
      $\mathbb{R}$
      &
      ??
      \\ \\ 
      use\_AH\_new\_smooth
      &
      $\mathbb{N}$
      &
      ??
   \end{tabular}
\end{table}

\newpage
%=============================================================================
\section*{Excision surface parameters}

\begin{table}[h]
   \centering 
   \begin{tabular}{ccc}
      Parameter  & Allowed values & Purpose \\
      \midrule\midrule
      do\_ex 
      &
      $\left\{0,1\right\}$
      &
      (Do not) excise
      \\ \\
      ex 
      &
      $\mathbb{R}_+$
      &
      ??
      \\ \\
      ex\_rmin 
      & 
      $\mathbb{R}_+$
      &
      Minimum radius to excise
      \\ \\
      ex\_rbuf\_reset
      &
      $\left\{0,1\right\}$
      &
      (Do not) reset excision buffers.
      \\
      \\ \\
      ex\_rbuf\{\_a\}\{\_2,3\}
      &
      $\mathbb{R}_+$
      &
      Excision buffer inside inside apparent
      \\
      & & 
      as a fraction of the apprent horizon radius.
      For black hole 1,2,3.
      \\
      & &
      \_a is the buffer after the first time step.
      \\ \\
      ex\_reset\_rbuf 
      &
      $\mathbb{N}$
      &
      ??
      \\ \\
      ex\_max\_repop 
      &
      $\mathbb{N}$
      &
      ??
      \\ \\
      ex\_repop\_io
      &
      $\mathbb{N}$
      &
      ??
      \\ \\
      ex\_dphi\_repop
      &
      $\mathbb{N}$
      &
      ??
      \\ \\
      ex\_relative\_rbuf
      &
      $\mathbb{N}$
      &
      ??
      \\ \\
      ex\_max\_rbuf
      &
      $\mathbb{R}_+$
      &
      ??
      \\ \\
      ex\_freeze\_r\_omt
      &
      $\mathbb{N}$
      &
      ??
   \end{tabular}
\end{table}

\newpage
%=============================================================================
\section*{General initial data parameters}
\begin{table}[h]
   \centering 
   \begin{tabular}{ccc}
      Parameter  & Allowed values & Purpose \\
      \midrule\midrule
      use\_harm\_kerr
      &
      $\left\{0,1\right\}$
      &
      Whether or not to use harmonic coordinates for black holes.
   \end{tabular}
\end{table}

\newpage
%=============================================================================
\section*{TwoPunctures definitions}

For use with the Standalone TwoPunctures code,
see \href{https://github.com/JLRipley314/Standalone-TwoPunctures-C-Cpp}{here}.

\begin{table}[h]
   \centering 
   \begin{tabular}{ccc}
      Parameter  & Allowed values & Purpose \\
     \midrule\midrule
      TP\_use\_twopunctures 
      & \{0,1\} & (Do not) use 
      \\ \\
      TP\_points\_A
      & $\mathbb{N}$ 
      & Number of collocation points
      \\ \\
      TP\_points\_B
      & $\mathbb{N}$ 
      & Number of collocation points
      \\ \\
      TP\_points\_phi 
      & $\left\{4n, n\in\mathbb{N}\right\}$ 
      & Number of collocation points in phi direction
      \\ \\
      TP\_lapse\_kind 
      & $\mathbb{N}$ 
      & Kind of initial lapse used; see source code 
      \\ \\
      TP\_offset\_\{plus,minus\} 
      & $\mathbb{R}$ 
      & Location (in uncompactified space) of the
      plus/minus black hole
      \\ \\
      TP\_target\_M\_\{plus,minus\} 
      & $\mathbb{R}_+$ 
      & Target mass of the plus/minus black hole
      \\ \\
      TP\_par\_P\_\{plus,minus\}\_\{x,y,z\} 
      & $\mathbb{R}$ 
      & Initial momentum vector  
      \\ \\
      TP\_par\_S\_\{plus,minus\}\_\{x,y,z\} 
      & $\mathbb{R}$ 
      & Initial spin vector 
      \\ \\
   \end{tabular}
\end{table}

\newpage
%=============================================================================
\section*{Einstein scalar Gauss-Bonnet parameters}
See \cite{East:2020hgw}.
\begin{table}[h]
   \centering 
   \begin{tabular}{ccc}
      Parameter  & Allowed values & Purpose \\
      \midrule\midrule
      csf\_on 
      &
      $\left\{0,1\right\}$
      &
      (Do not) evolve scalar field
      \\ \\
      lambda\_mg 
      &
      $\mathbb{R}$
      &
      scalar GB coupling $\lambda$ in $\lambda\phi\mathcal{G}$
      \\ \\
      ts\_mg 
      &
      $\mathbb{R}_+$
      &
      Time to turn on scalar Gauss-Bonnet couplings as a funtion of $M$.
      \\ \\
      \{ts,te\}\_damp\_mg 
      &
      $\mathbb{R}_+$
      &
      Time to start reducing coupling parameter.
      \\ \\
      damp\_mg\_r 
      &
      Vector, $\left[
         \mathbb{R}_+ \; \mathbb{R}_+ \; \mathbb{R}_+
      \right]$
      &
      Radii of ellipsoid region (in compactified coordinates) 
      \\
      & & 
      where coupling is reduced.
      \\ \\
      damp\_mg\_rbuf 
      &
      $\mathbb{R}_+$
      &
      Fractional radius over which coupling goes to zero,
      \\
      & & 
      i.e. coupling decreases from full value to zero in the outer 
      \\
      & & 
      buffer region making up this fraction of the radius of the above region.
   \end{tabular}
\end{table}
%=============================================================================
\bibliographystyle{unsrt}
\bibliography{thebib}
%=============================================================================
\end{document}
